\documentclass[prl,twocolumn,superscriptaddress,preprintnumbers,floatfix]{revtex4-1}

\usepackage{etex}
\usepackage{ifpdf}
\usepackage{hyperref}
\usepackage{dcolumn}
\usepackage{url}
\usepackage{amsmath}
\usepackage{amscd}
\usepackage{amsfonts}
\usepackage{amssymb}
\usepackage{bm}   % bold math
\usepackage{bbm}
\usepackage{verbatim}
\usepackage{stmaryrd}
\usepackage{amsthm}
\usepackage{xcolor}
\usepackage{setspace}
\usepackage{cleveref}
\usepackage{graphicx}

\usepackage{tikz}
\usetikzlibrary{arrows}
\usetikzlibrary{automata}
\usetikzlibrary{decorations.pathreplacing}
\usetikzlibrary{positioning}
\usetikzlibrary{plotmarks}
\usetikzlibrary{calc}
\usetikzlibrary{patterns}
\usetikzlibrary{external}
\tikzexternalize
\tikzsetexternalprefix{images/}
\usepackage{pgfplots}
\pgfplotsset{compat=newest}

\usepackage{xspace}

% information theory
\renewcommand{\H}[1]    {\ensuremath{\operatorname{H}\left[#1\right]}}
\newcommand{\I}[1]    {\ensuremath{\operatorname{I}\left[#1\right]}}

% processes
\newcommand{\Language}  {\ensuremath{\mathcal{L}}\xspace}
\newcommand{\Process}   {\ensuremath{\mathcal{P}}\xspace}

\newcommand{\leftword}  {\overleftarrow{w}}
\newcommand{\rightword} {\overrightarrow{x}}

\newcommand{\emachine}        {$\epsilon$-machine\xspace}
\newcommand{\etomata}   {$\epsilon$tomata\xspace}


\begin{document}

\title{Mathematical Foundations of Computational Mechanics}

\author{Ryan G. James}
\email{ryan.james@colorado.edu}
\affiliation{Department of Computer Science, University of Colorado at Boulder, Boulder, Colorado 80309}

\author{Nicole F. Sanderson}
\email{nicole.sanderson@colorado.edu}
\affiliation{Department of Mathematics, University of Colorado at Boulder, Boulder, Colorado 80309}

\date{\today}
\bibliographystyle{unsrt}

\date{\today}

\begin{abstract}

We make rigorous the extension of formal language theory to processes and further extend the theory to include atoms. 

\vspace{0.1in}
\noindent {\bf Keywords}:

\end{abstract}

\maketitle


\section{Notes}
Formal language theory utilizes the notions of (left) quotients, (right) quotients, right languages, and left languages to characterize languages.  Our goal is to rigorously define analogous concepts in the realm of processes.  A language \Language is defined to be a subset of the free monoid $\Sigma^{\star}$on an alphabet $\Sigma$. The (left) quotient of a language \Language for a word $w$ is defined to be $w^{-1}\Language = \{ x | wx \in \Language \}$.

A process \Process is defined as 


\bibliographystyle{unsrt}
\bibliography{bibliography}

\end{document}
